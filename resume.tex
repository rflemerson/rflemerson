\documentclass[a4paper,10pt]{article}
\usepackage[margin=0.5in,nofoot]{geometry}
\usepackage{fontawesome5}
\usepackage{hyperref}
\usepackage{titlesec}
\usepackage{xcolor}

\hypersetup{
    colorlinks=true,
    linkcolor=black,
    filecolor=black,
    urlcolor=black,
    citecolor=black
}

\titleformat{\section}{\large\bfseries}{\thesection}{1em}{}[\titlerule]
\titlespacing*{\section}{0pt}{*5}{*1}

\newcommand{\entry}[4]{
    \noindent\textbf{#1} \hfill #2 \\
    \noindent\textit{#3} \hfill \textit{#4} \\
    \vspace{2pt}
}

\begin{document}

\pagenumbering{gobble}

\noindent
\begin{minipage}[t]{0.5\textwidth}
    \textbf{\LARGE Rafael Emerson Leal} \\[0.6em]
    {\large \color{darkgray} Desenvolvedor Full-Stack}
    \vspace{0.4em}
\end{minipage}%
\begin{minipage}[t]{0.5\textwidth}
\raggedleft
\href{https://maps.app.goo.gl/fPieLVkxevk84gFT7}{\faMapMarker \space Curitiba -- PR, Brasil}\\
\href{tel:+5541995717120}{\faPhone \space +55 (41) 99571-7120}
\quad
\href{mailto:rflemerson@gmail.com}{\faEnvelope \space rflemerson@gmail.com}\\
\vspace{0.2em}
\href{https://github.com/rflemerson}{\faGithub \space rflemerson} \quad
\href{https://www.linkedin.com/in/rflemerson/}{\faLinkedin \space rflemerson}
\end{minipage}

\vspace{1em}

\section*{Objetivo}\vspace{0.6em}

Busco uma oportunidade de nível inicial/júnior na área de tecnologia, onde eu possa aplicar meus conhecimentos em desenvolvimento de software. Tenho interesse em contribuir com novos projetos e aprimorar minhas habilidades técnicas em um ambiente que incentive o aprendizado contínuo.

\section*{Resumo Profissional}\vspace{0.6em}

Desenvolvedor full-stack com experiência backend e frontend com frameworks como Flask, Django e Angular. Conhecimento em pipelines de CI/CD (Integração e Entrega Contínuas) e experiência com Docker para containerização de aplicações e Kubernetes para orquestração de contêineres. 

\vspace{0.6em}\noindent
Graduando em Engenharia Elétrica com ênfase em Sistemas Eletrônicos Embarcados pela UFPR, tenho afinidade por sistemas de baixo nível, especialmente em sistemas operacionais e na interação entre hardware e software.

\vspace{0.6em}

\section*{Formação Acadêmica}\vspace{0.6em}

\entry{UFPR \textbullet{} Universidade Federal do Paraná}{\faCalendar\space jan. 2020 -- dez. 2025}{Engenharia Elétrica com ênfase em Sistemas Eletrônicos Embarcados}{Curitiba -- PR}
\vspace{-0.6em}

\section*{Experiência}\vspace{0.6em}

\entry{CITS \textbullet{} Centro Internacional de Tecnologia de Software}{\faCalendar\space mar. 2023 -- atual}{Desenvolvimento de Software -- Estagiário}{Curitiba -- PR}\vspace{-1.6em}
\begin{itemize}
    \setlength\itemsep{0em}
    \item Desenvolvimento full stack com Flask/Django (Python) e Angular (TypeScript/HTML5/SCSS)
    \item Implementação de Single Sign-On (SSO) via Microsoft Entra ID (ex-Azure AD) e LDAP para sistemas corporativos
    \item Criação de pipelines CI/CD com GitLab CI/Azure DevOps, integrando SAST e DAST
    \item Construção de imagens Docker seguindo as melhores práticas
    \item Desenvolvimento de manifests YAML para orquestração Kubernetes (AKS/EKS)
\end{itemize}

\entry{Lactec}{\faCalendar\space fev. 2020 -- mar. 2023}{Pesquisa e Desenvolvimento -- Bolsista ITI}{Curitiba -- PR}\vspace{-1.6em}
\begin{itemize}
    \setlength\itemsep{0em}
    \item Desenvolvimento de algoritmos de visão computacional e processamento de dados em 2D/3D
    \item Configuração de ambientes de desenvolvimento e contêineres voltados para o treinamento de modelos de aprendizado de máquina (ML)
    \item Preparação e pré-processamento de conjuntos de dados, com etapas de limpeza, transformação e rotulagem
\end{itemize}

\section*{Habilidades Técnicas}\vspace{0.6em}

\begin{itemize}
    \setlength\itemsep{0em}
    \item \textbf{Back-end}: Python (Django, Flask)
    \item \textbf{Front-end}: Angular (TypeScript, HTML, CSS/SCSS), Angular Material, Bootstrap
    \item \textbf{Bancos de Dados}: SQL (PostgreSQL, SQLite)
    \item \textbf{Sistemas e Hardware}: C/C++, VHDL
    \item \textbf{DevOps}: Docker, Kubernetes, GitLab CI/CD, Azure DevOps
    \item \textbf{Outras ferramentas}: GNU/Linux, Bash, Git
\end{itemize}

\end{document}
