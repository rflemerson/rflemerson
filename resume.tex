\documentclass[a4paper,10pt]{article}
\usepackage[margin=0.5in,nofoot]{geometry}
\usepackage{fontawesome5}
\usepackage{hyperref}
\usepackage{titlesec}
\usepackage{xcolor}

\hypersetup{
    colorlinks=true,
    linkcolor=black,
    filecolor=black,
    urlcolor=black,
    citecolor=black
}

\titleformat{\section}{\large\bfseries}{\thesection}{1em}{}[\titlerule]
\titlespacing*{\section}{0pt}{*1}{*1}

\newcommand{\entry}[4]{
    \noindent\textbf{#1} \hfill #2 \\
    \noindent\textit{#3} \hfill \textit{#4} \\
    \vspace{2pt}
}

\begin{document}

\pagenumbering{gobble}

\noindent
\begin{minipage}[t]{0.5\textwidth}
    \textbf{\Large Rafael E. Leal}
    
    \vspace{0.4em}
\end{minipage}%
\begin{minipage}[t]{0.5\textwidth}
    \raggedleft Curitiba -- PR
    
    {\color{blue}} \href{tel:+5541995717120}{\faPhone \space (41) 99571-7120}
    {\color{blue}} \href{mailto:rflemerson@gmail.com}{\faEnvelope \space rflemerson@gmail.com}
    
    \vspace{0.2em}
    \quad
    {\color{blue}} \href{https://github.com/rflemerson}{\faGithub \space GitHub} \quad
    {\color{blue}} \href{https://www.linkedin.com/in/rflmerson/}{\faLinkedin \space LinkedIn} \\
\end{minipage}

\vspace{1em}

\begin{center}
    \textbf{\Large Desenvolvedor Full Stack}
\end{center}
\vspace{0.5em}

\section*{Resumo Profissional}
\vspace{0.6em}

{Graduando em Engenharia Elétrica com ênfase em Sistemas Eletrônicos Embarcados pela UFPR. Desenvolvedor Full Stack com experiência em frameworks como Flask, Django e Angular. Conhecimento em práticas de CI/CD para integração e entrega contínua. Experiência com Docker para conteinerização de aplicações e Kubernetes para orquestração de contêineres em ambientes de desenvolvimento e produção.}

\vspace{0.6em}

\section*{Habilidades Técnicas}
\vspace{0.6em}

\subsection*{Linguagens de Programação e Frameworks}
\begin{itemize}
    \setlength\itemsep{0em}
    \item \textbf{Front-end}: Angular (TypeScript, HTML, CSS/SCSS), Angular Material, Bootstrap
    \item \textbf{Back-end}: Python (Django, Flask), SQL (PostgreSQL, SQLite)
    \item \textbf{Sistemas e Hardware}: C/C++, VHDL
\end{itemize}

\subsection*{DevOps e Infraestrutura}
\begin{itemize}
    \setlength\itemsep{0em}
    \item \textbf{DevOps}: Docker, Kubernetes, GitLab CI/CD, Azure DevOps
    \item \textbf{Ferramentas}: GNU/Linux, Bash, Git
\end{itemize}

\section*{Experiência}
\vspace{0.6em}

\entry{CITS \textbullet{} Centro Internacional de Tecnologia de Software}{\faCalendar \space Mar/2023 -- Presente}{Desenvolvimento de Software -- Estagiário}{Curitiba -- PR}
\vspace{-1.6em}
\begin{itemize}
    \setlength\itemsep{0em}
    \item Desenvolvimento full stack com Flask/Django (Python) e Angular (TypeScript/HTML5/SCSS)
    \item Implementação de SSO via Microsoft Entra ID (ex-Azure AD) e LDAP para sistemas corporativos
    \item Criação de pipelines CI/CD com GitLab CI/Azure DevOps, integrando SAST e DAST
    \item Construção de imagens Docker seguindo as melhores práticas
    \item Desenvolvimento de manifests YAML para orquestração Kubernetes (AKS/EKS)
\end{itemize}

\entry{Lactec}{\faCalendar \space Fev/2020 -- Mar/2023}{Pesquisa e Desenvolvimento -- Bolsista ITI}{Curitiba -- PR}
\vspace{-1.6em}
\begin{itemize}
    \setlength\itemsep{0em}
    \item Desenvolvimento de algoritmos de visão computacional e processamento de dados em 2D/3D
    \item Configuração de ambientes de desenvolvimento e contêineres voltados para o treinamento de modelos de machine learning (ML)
    \item Preparação e pré-processamento de datasets, com etapas de limpeza, transformação e criação de labels
\end{itemize}

\section*{Formação Acadêmica}
\vspace{0.6em}

\entry{UFPR \textbullet{} Universidade Federal do Paraná}{\faCalendar \space Jan/2020 -- Dez/2025}{Engenharia Elétrica com ênfase em Sistemas Eletrônicos Embarcados}{Curitiba -- PR}

\section*{Certificações}
\vspace{0.6em}
\entry{ABNT \textbullet{} Associação Brasileira de Normas Técnicas}{\faCalendar \space Jul/2024}{Inteligência Artificial aplicada à Melhoria de Processos Organizacionais}{}

\end{document}
